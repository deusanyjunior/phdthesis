%% ------------------------------------------------------------------------- %%
\chapter{Conclusions}
\label{cap:conclusions}



 \epigraph{
	We are human \\
	After all. \\
	Flesh in common \\
	After all.}
 {Daft Punk}

% contribuições para a ciência

% citar fog e edge computing?!

Mobile Music is a growing field that can still be considered in its initial stages of experimentation.
Mobile technologies are improving and should continue to do so as mobile computing remains the focus of the market.
Musicians may take advantage of these novelties within their compositional processes, and will benefit the most if they become aware of the constraints inherent to these technologies.
This work thus proposed to evaluate these technologies and present diverse evaluations as well as applications as references for those who want to explore mobile music possibilities and experiment with appealing solutions for large scale user interaction, collaboration, and cooperation. 

Delay between messages is an important parameter which was evaluated in the experiments here presented, with reasonable results that reveal the dependence of the statistics on the delay.
Message loss may have occurred in our evaluations due to overhead on the network buffers:
in the event that at least two packets are grouped by a network buffer before being sent, it is possible to have more than 10 messages per second even when using 150~ms delays between packets, which would be a problem for Cloud Services such as Pusher, but would work flawlessly for alternatives such as PubNub or Unicast.

When using Cloud Services it is possible that paid plans would offer different clusters that could be selected to optimize routes between performers and users of a mobile music application.
Although clusters are available around the world for some Cloud Services, there are occasions when it may be necessary to try a different cloud service for better cluster localization and routing.

The cloud services used during this research had some specific settings depending on the plan. 
The cloud services free account was used most of the time but a paid plan was setup in order to compare the services during some evaluations.
In this case, I had to use two different accounts and two pair of keys for each cloud service evaluated.
The account set up and configuration were available through the user interface at the cloud service website.
The free account is ready to use at the time of registering an account at the web site of each cloud service.
After paying for the paid plan with the credit card, the new settings and limits become available automatically

Results obtained from all evaluations suggest that the mobile music community may use Cloud Services as an option for intercommunication in music applications, an alternative that facilitates implementation and improves scalabity, overcoming technological constraints faced by previous mobile music experiments restricted to small groups of participants.
The delay and reliability of the service has proven to be suitable for present technologically-demanding applications, and we may expect improvements in a near future broadnening the range of possible applications.
Push-notifications help users to send and receive data from any part of the world and control the final sound from their own mobile devices, using other users' inputs to influence their music creation in a collaborative way.
In this way, many musical works formerly conceived for local networks might now be extended to cloud-based services.

The evaluations using the academic network provided insights into many challenges and difficulties that may be faced by those who plan to use it.
The initial configuration of the network and the requests for changing settings usually depend on many agents and may take from a few weeks to months or even years (in the case of IPv6 and Multicast).
Although the throughput is much better in Unicast (and probably would be in Multicast) than using Cloud Services, the burden of the network setup may discourage all but the most tenacious mobile musicians.
Network services such as IPv6 and Multicast were found to be still unstable within the academic network, despite their great theoretical advantages, and Multicast could not even be adequately tested, due to years of recurring problems with network settings over the full network path, inbound and outbound.


\paragraphdesc{academic network settings}
It is important to notice that in order to use the academic network I had to contact many network managers from time to time.
The RNP managers requested me a signed document assuming the responsibilities regarding any problem caused to the network during my evaluations.
This document had to be signed by my advisor and they started managing the requested settings after receiving the document.
The UMich network managers set-up everything considering my UMich number and email in order to register the requests under my responsibilities.

\paragraphdesc{academic network set up}
The settings requested during this research at both Universidade de São Paulo and University of Michigan was a VLAN with IPv4 and IPv6 blocks for interaction through Unicast and Multicast services, and also with Internet access for intercommunication with the Cloud Services.
Although it seemed to be an easy request from my point of view, the services were unstable during most of the time of the research.
The Multicast was the most troublesome setting and required many requests to Network Operations Center~(NOC) throughout the whole path between Universidade de São Paulo and University of Michigan.
The Unicast and Internet were available most of the time, on the other hand.

The applications developed during this research were heavily used and are still under constant development and improvement.
Users around the world are still requesting new features, specially in the case of Sensors2PD, Sensors2OSC, SuperCopair, and Crowd in C[loud], which can be taken to mean that these applications have continued to attract some attention and interest even though their original performances and/or developing scenarios are long gone.
Their code will remain open source in order to allow users to learn from these technologies and developers to collaborate and improve them, participating in the source code projects as members, which is an important aspect of the free open source software community.

\section{Published papers}

During the development of this research, many papers were published in order to help disseminate its results.
A description of the international papers related to this thesis is given below.

The paper ``FFT benchmark on Android devices: Java versus JNI''~\citep{deCarvalhoJunior2013fftbenchmark} discusses the effects using Java or JNI for DSP on Android devices.
The results show that the programmer can use low level and parallel programming depending depending on hardware and OS.
These options have better performance under specific circumstances, such as multi-core devices, newer OS, and larger block size.
The full paper is presented at Appendix~\ref{ape:papersbcm2013}.

The discussion presented in the paper named ``Touches on the line: Sharing Csound scores using web server and mobile phones''~\citep{deCarvalhoJunior2013touches} describes the possibilities of using a RoR web server in order to exchange notes or scores through the Internet.
A web server created with RoR requires few commands and a single Ruby file.
This proposal facilitates the development of interactive mobile performances for local and distributed network.
The full paper is presented at Appendix~\ref{ape:papericsc2013}.

``Notes on the Elimination of the Mobile Music Audience''~\citep{Bandeira2014notes} is an inspiring paper that presents a discussion about new kind of performance without barriers and breaking with traditional art concepts.
It is inspired by the text ``Notes on the Elimination of the Audience'' by Allan Kaprow.
The full paper is presented at Appendix~\ref{ape:paperbats2014}.

The first paper about Sensors2 applications was ``Sensors2PD: Mobile sensors and WiFi information as input for Pure Data''~\citep{deCarvalhoJunior2014sensors2pd},
although the paper discussed in the previous paragraph used Sensors2PD initial code.
This paper discuss the many possibilities of using Sensors2PD to interact with Pure Data patches on Android devices through this application.
All sensors available in a device can act as an actuator and send data to receivers created on the patch.
The full paper is presented at Appendix~\ref{ape:papericmc2014}.

``Indoor localization during installations using Wi-Fi''~\citep{deCarvalhoJunior2015indoor} presents the approach of using WiFi signal quality as an input for Sensors2PD during the installation ``Hoketus''.
The discussion emphasizes the advantage of using WiFi instead of GPS for indoor localization at installations and performances.
GPS signal is noisy or unavailable inside buildings with concrete structure and also has interference from the environment nature.
On the other hand, WiFi signal can be disposed depending on user needs through many devices like routers, or smartphones hot-spots. 
The latter option was used during the installation.
The full paper is presented at Appendix~\ref{ape:papernime2015}.

``Sensors2OSC''~\citep{deCarvalhoJunior2015sensors2osc} discuss the approach to conceive this application and all its features.
Sensors2OSC allows users to send samples from Android sensors through OSC inside a network.
The application follows the same idea of Sensors2PD where the transmission of samples of a sensor is activated/deactivated in real-time, but includes the possibility of IP and network port definition for routing the se samples.
The application and the paper came from a partnership with Thomas Mayer.
The full paper is presented at Appendix~\ref{ape:papersmc2015}.


``Computer Music through the Cloud: Evaluating a Cloud Service for Collaborative Computer Music Applications''~\citep{deCarvalhoJunior2015computer} is a paper that summarize the first attempts to used Cloud Service for mobile music communication during this research.
The paper shows results with data transmission through Pusher Cloud Service between Brazil and USA.
The evaluation considered data with different sizes and gave insights to the advantage of using this service at an unusual situation.
The full paper is presented at Appendix~\ref{ape:papericmc2015}.

The Pusher Cloud Service is also evaluated at the paper ``SuperCopair: Collaborative Live Coding on SuperCollider through the cloud''~\citep{deCarvalhoJunior2015supercopair}.
In this case, the service is evaluated under unusual circumstances by means of an application for pair-programming live coding.
SuperCopair application allows users to collaborate inside a live coding session using the SuperCollider language, share the code, and share the code synthesis through the Internet.
The burden to initiate a collaborative live coding session is surpassed by one or two shortcuts inside Atom.io IDE.
The sound synthesis happens inside the IDE while the code is shared through the Cloud Service.
The full paper is presented at Appendix~\ref{ape:papericlc2015}.

``Cooperative Live Coding as an instructional model''~\citep{deCarvalhoJunior2015cooperative} discuss the experiences of using SuperCopair with users from different cities and countries in real-time.
The idea of cooperation is highlighted because the users helped themselves fixing codes and writing comments with suggestions close to friends code. during the live sessions.
The full paper is presented at Appendix~\ref{ape:paperclei2015}.

``Crowd in C[loud]: Audience Participation Music with Online Dating Metaphor using Cloud Service''~\citep{Lee2016crowd} is paper about an experience with Cloud Services in Mobile Music.
The piece ``Crowd in C[loud]'' took advantage of Web Audio technology for sound synthesis through the browser of mobile phones and allowed an online dating of melodies using Cloud Services for network interaction in real-time during a live performance inside a theater.
Around 60 participants joined the session which last for 10 minutes.
An important point about this piece is that the audio came only from participants devices but could be heard by the whole audience.
The full paper is presented at Appendix~\ref{ape:paperwac2016}.

A discussion about the network communication is presented in ``Understanding Cloud Service in the Audience Music Performance of Crowd in C[loud]''~\citep{deCarvalhoJunior2016understanding}.
During the performance we had a backup computer connected to the Cloud Service monitoring the network communication in all channels created on the Cloud Service.
Data consumption and traffic is presented as motivation for using Cloud technologies in performances due to its capacity and scalability.
The full paper is presented at Appendix~\ref{ape:papernime2016}.

``Open band: Audience Creative Participation Using Audio Synthesis''~\citep{Stolfi2017openwac} is another performance/installation using Web Audio for sound synthesis including audience interaction through the network.
The communication can occur within local network or Cloud Services, depending on the proposal.
This paper discuss the technical aspects and advantages of using Web Audio synthesis for this performance.
The full paper is presented at Appendix~\ref{ape:paperwac2017}.

``Open Band: A Platform for Collective Sound Dialogues''~\citep{Stolfi2017openam} discuss Open Band performance in terms of the user experience during the performances.
Local and distributed performances favored distinct evaluations regarding the same application.
The full paper is presented at Appendix~\ref{ape:paperam2017}.

