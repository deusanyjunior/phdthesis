%% ------------------------------------------------------------------------- %%
\chapter{Mobile Networks}
\label{cap:mobilenetworks}

% \epigraph{
% Around the world,\\
% Around the world,\\
% Around the world,\\
% Around the world.}
% {Daft Punk}

\paragraphdesc{evolution of network technologies}
The communication concept reached the current millennium with many small devices communicating through wireless broadband.
This concept extends the possibilities as diverse interfaces can now be attached to anything providing communication with devices anywhere in the world.
%%FEULO achjo que não saquei bem essa utima frase, ficou confusa
On the other hand, the communication medium used may physically define the throughput based on routes and the amount of information for exchanged data.
In terms of computer networks, although it is possible to use a wired connection, mobile devices are mostly free from cables to favor its own mobility aspect and they require batteries to maintain this condition.
These two points are important for mobile networks due to the restrictions imposed by them:
%%FEULO ..networks due the restrictions they impose.
wireless connections are subjected to noise more than wired ones and power-saving strategies allow longer data transmission while avoiding communication breakdown by full discharging.
%%FEULO ...data transmission avoiding communication breakdown caused by full discharging.
Some details regarding computer networks will be discussed in the following paragraphs and sections considering mobile devices communication and connection with other devices. 
%% .. connection with other device types
\paragraphdesc{mobile network technologies and 'always on' concept}
While WiFi networks depend on wireless routers that cover tens of meters with Internet access, the mobile networks cover tens of kilometers from the base station.
%% FEULO, é "always on" mesmo? always connected me soou melhor
As the culture of being "always on" is getting more prevalent, mobile networks demand range and quality to be increased all the time due to the constant data traffic necessity requested by users.
%% FEULO prevalent, range and quality of mobile networks need to be increased constantly due the user's data traffic necessity 
When the mobile network is unstable or presents low quality, switching from mobile networks to WiFi networks is thus the option, even if the user needs to stop physically somewhere just to get online, losing their mobility in exchange for Internet access and that "always on" feeling.
However, it is important to notice that WiFi networks normally present higher throughput than mobile networks and allow for use at places in which mobile networks may be unsupported, such as inside concrete buildings that would prevent the direct passage of a magnetic field from a mobile network antenna.

\paragraphdesc{physical layer and link layer}
In order to take advantage of WiFi or mobile network connections, a device needs a specific network interface to allow communication through a physical medium and is part of the physical layer of any network.
Mobile devices intercommunicate through electromagnetic fields over the air or space with their base stations (or routers), while the wired connection from the base station is made through any other medium, such as fiber optics.
Infra Red, Bluetooth, and USB interfaces can also work as physical
%%FEulo USB interfaces that can also
interfaces for mobile devices as these interfaces can be used to connect to a router device connected to a network, e.g., a computer sharing its Internet through the USB port with a phone.
Depending on the interface, different protocols are used to exchange the bits between the interfaces, but the link layer is responsible for deciding how to pack the data depending on the interface available (and connected to a network) when the network layer wants to send this data without knowing the physical layer.
When receiving data from a physical layer, the link layer verifies if the data needs to be resent through the physical layer or routed to the network layer.

\paragraphdesc{network layer}
Packets with data or partial data are moved from one network layer to another following rules that depend on the destination address and routing algorithms.
Many services available on the network layer are used to transport the packets in order to deliver the following rules defined from the transport layer.
Guaranteed delivery is one of the services available at the scope of the network layer, and the IPv4 or IPv6 address works as a guide to define the route taken by a packet forwarded inside a network.
Conceived during the 1990's, the IPv6 can help to correlate a single network address to a single device, as it used to be during the onset of IPv4 implementation, and then facilitate the transport of packets.

\paragraphdesc{transport layer}
In the ambit of the transport layer, solutions for exchanging data on mobile networks may have to opt for solutions based on unreliable connections.
The transmission rate varies while a device is moving depending on the signal quality, the distance from antennas, and communication interferences.
Although unreliable connections inspire unreliable transport protocols such as UDP, the TCP can also be used, but requires a persistent connection in order to avoid reconnection for every data exchange.
The main idea inside this layer is that there is a logical interconnection between two hosts where one or both can be using a mobile network, but none of them needs to know their actual interconnection path.

\paragraphdesc{application layer}
Mobile network applications assume the device will often change its IPv4 address, disconnect for a long period, and can discharge while connected.
These circumstances require sessions to keep devices logged independently of a network address, data synchronization based on online state, and synchronization methods that download only important data from time to time or consider downloading heavy data online while connected through WiFi in order to save money on a mobile network data plan.
In this case, dealing with unstable connections is a situation that requires good communication strategies from the application layer point of view.
The protocols and interfaces used may thus interfere with the communication depending on the purpose of data exchange.
If every single data is important, the devices may require a confirmation for each message sent, for example.
On the other hand, the devices can just send as many messages as possible using protocols that are fast and unreliable if data loss is expected, planned, or desired.
Network application programmers and users need to adhere to conditions regarding data transmission between devices, as packet delay and packet loss are inherent to this context.
Notwithstanding these network constraints, dealing with them is an option.
%% FEULO pra mim essa ultima frase não fez sentido =P
\paragraphdesc{data loss or data delay?! a touch of information theory}
In terms of communication, when a person decides to send a message, this person needs to decide the message encoding, the transmitter, the channel, and expects that the destination will share the same channel, use a receiver, and decode the message correctly.
This concept is based on Shannon's Information Theory~\citep{Shannon1948mathematical}, that is widely used on communication studies~\citep{Fiske2010introduction}.
As the receiver may need to answer a message, probably the same channel and encoding will be used for communication to occur.
Shannon also studied this scenario~\citep{Shannon1961two}, and the networks follow some of their results even today.
Encoding a message effectively instead of sending the complete raw information has its effect as the channel will need to move less data.
On the other hand, the channel selected may delay the message delivery or impose noise risk, so that the message may be delivered when it is not necessary or even is not delivered.
%%FEULO or even not delivered at all
The Information Theory was an important key for Internet development, and selecting the best network options for data exchange based on this theory is a desirable expectation for people who deal with network communication, or even any kind of communication approach.

\paragraphdesc{describe the chapter}
This chapter discusses mobile network structure from devices to services.
The devices used for communication on mobile networks are presented with a focus on mobility, power consumption, network constraints, and data exchange.
Interfaces are presented with technical information for users intending to apply them, while the topologies formed using these interfaces are shown in correlation with classic computer music approaches.
The protocols and services used in this thesis are also discussed in terms of their advantages and disadvantages, including specific constraints of some services chosen for this work specifically.

%% ------------------------------------------------------------------------- %%
\section{Mobile Devices}
\label{sec:mobiledevices}

\paragraphdesc{mobility and its importance}
Computing data everywhere while moving is just one of the advantages of mobile devices.
In the past, vehicles were used for transporting computers powered by batteries or power generators, however, the mobility got attention with the size reduction of components and devices.
The mobile devices achieve their place 
%%FEULO não ei se "place" é a melhor palavra pra usar aqui, eu tentaria status ou algo similar
with the size reduction of the components reaching nanometers nowadays, that are billionths of a meter.
%%FEULO Vc repetiu "size reductions of components" duas vezes muito próximas, acho q tb não precisa falar que nanometro é um bilionesico de metro.
This size reduction improved the capacity of mobile devices allowing more components to fit in a handheld board, with diverse communication interfaces, gigabytes of data, many sensors, and a long life battery.
Although the devices can use all of these technologies in order to exchange data, the users of these devices face many issues related with
%%FEULO related to
battery and power consumption, data plan costs, handover, and data sync.

\paragraphdesc{problems: battery and power consumption discussion}
Choosing the smartphone to buy can be a life changing
%%FEULO a life changer ou life changing, mas se a idéia é expor a complexidade da escolha eu usaria outra expressão
as the user needs to select brand, the features, the mobile network, the sensors, and all the other features that are available on the market options.
The smartphones are constantly updated with new features since their conception and the impossibility of using many features at a time due to power consumption encouraged the improvement of the batteries technology.
%%FEULO Since their conception, Smartphones were/have been constantly updated and the impossibility of using the increasing number of features at a time...
Power banks born in 2001, but they became a must-have accessory for smartphones only in later years.
%%FEULO acho que born fica mto comercial, tenta were developed
%% ...only years later.
The batteries are also improving, but as the smartphones are substituting personal
%%FEULO avho que aqui caberia "taking places", substituting da ideia de algo temporário
computers and notebooks --- devices there are utilized with a power supply most of the time ---, the smartphones are expected to work like them in terms of availability, requiring better batteries and recharging technologies.

\paragraphdesc{mobile networks cost: company and consumer}
Internet access is another requirement for most of users of smartphones.
WiFi and mobile network
%%FEULO networks
are available at multiple places and while the former is faster, the latter works as an alternative for situations where the first is unavailable or there is need of moving from one place to another connected to the Internet.
%%FEULO ...another while connected to the internet
The idea of using mobile networks as an alternative for WiFi is also related to the cost for data consumption on mobile network plans.
Even if it is part of the business model of the mobile network service providers, the cost affects both consumers and companies.
Consumers will manage their data plan throughout the month in order to save money but they will change the network providers once their connection is not reliable when and where they need Internet access.
Companies offer different data plans that can be affordable for attracting users, but as the cost for installing a new antenna is higher, some companies are also investing in public WiFi hotspots for consumers that have an unreliable connection at some places, such as airports or shopping malls.
A qualitative and performance comparison between these two technologies are presented at \cite{Lehr2003wireless} and \cite{Gass20103gwificomparison}, respectively.

\paragraphdesc{handover and data sync}
Smartphones are expected to have battery charged and Internet connection access all the time.
Users need to be synced with online data as fast as possible and data syncing requires real-time communication between the mobile device and many data servers.
Instead of checking the servers all the time, the devices may apply some solutions such as a keep-alive socket for receiving notifications regarding any new data.
Being connected all the time independently of the network interface or geographic location may result in switching between mobile network antennas or WiFi routers, or even between both mobile networks and WiFi technologies, additionally to recharging the device whenever possible.
Furthermore, dealing with these constraints of mobile devices is inherent in daily life of smartphone users. % and smartphones' OS and interfaces developers.



%% ------------------------------------------------------------------------- %%
\section{Network Interfaces}\index{mobile networks!network interfaces}
\label{sec:networkinterfaces}

\paragraphdesc{evolution of ways of exchanging data}
The way devices exchange messages has evolved throughout the time with diverse technologies being attached at the same device.
%%FEULO atached to ou  atached at one device
Communication interfaces are compatible with many different devices and the communication quality does depend on the medium condition and noise interference level.
From calling to high definition video conference, and from bits per second~(bps) over infrared to gigabits per second~(gbps) through WiFi, the mobile device users need to choose an interface to do some of these activities then so they can use WiFi to call someone or try Infrared to transmit a video conference.

\paragraphdesc{physical network interfaces}
Network interfaces differ in technical settings including transmission range and bandwidth.
Short range interfaces can be used to predispose interaction between devices close to each other and some examples are Infrared, Bluetooth, Near Field Communication~(NFC), and WiFi.
In case the user wants to communicate with a devices far away, there are long range interfaces for this such as WiFi and 4G (or other mobile network interface such as 3G or 2G), however short range interfaces can also be used with the addition of signal repeaters or router devices.
Following the Am, Fm, and television transmission methods, new wireless technologies such as mobile networks and WiFi routers connected to the Internet are examples of solutions for long range data transmission among mobile devices.
Data throughput and bandwidth varies according to interface communication standard with ranges varying from bits to gigabytes per second independently on the short or long range transmission characteristic.

\subsection*{Short range technologies} 

\paragraphdesc{short range interfaces descriptions}
Short range communication has the advantage of face to face interaction and avoid distance constraints related to any communication through a physical medium.
Data transmission and infrastructure are subjected to less issues regarding security as the devices will be close and the communication environment can be better monitored when all participants share the same place.
The communication types can defined as Mobile-to-mobile, Fixed-to-mobile, Infrastructure-to-mobile, and Infrastructure-to-fixed depending on the devices participating~\citep{Deicke201optical}.

A common example of short range interface is the Infrared.
Even a child can manage a remote control to change a channel on a television, while the communication is normally done by Infrared technology.
The data transmission had no standard before the first meeting of the Infrared Data Association~(IrDA) in 1993, when the first specification defined the communication as serial, half-duplex, and asynchronous in a range up to 1 meter with transmitter angle from 15 to 30 degrees, and the receiver angle up to 15 degrees.
The bandwidth would vary from 2,400~bps to 115,200~bps at the first specification, and nowadays the GigaIR reaches 1~Gbps and in several meters.

Infrared interface has the advantage of being cheaper and efficient.
The waves are invisible to human eyes and require low power for data transmission while the error rate as low as $10^{-9}$.
Infrared technology efforts are expecting new models that can be scaled up to 5~Gpbs and a new standard with 10 Gbps for broadcasting services~\citep{Deicke201optical}.
One disadvantage of Infrared communication is that the transmitter and the receiver need to be facing each other while they communicate.
This disadvantage also imply a single characteristic to Infrared due to its connectionless communication protocol: every device in the range of the transmitter will receive the data allowing a remote control to communicate with a TV and DVD at the same time if they first is over the second and share the same manufacturer Infrared standard.
%%FEULO the same time if the first is over the second and share the same manufacturer Infrared standard. 

Another wireless technology for short range communication is the Bluetooth~\citep{Bhagwat2001bluetooth}.
This license-free technology was conceived as a single-chip radio operating in 2.4~GHz ISM~(industrial, scientific, and medical) radio frequency band.
The first specification was released in 2001 with throughput range from 36.3 to 585.6~kbps, although the link speed is 1~Mbps.
The maximum distance supported for this specification is 100~meters though the higher the distance the higher is the power consumption~\citep[p.~20]{Gupta2013insidebluetooth}, and the recommended distance is 10~m in the end.
Current specification provides possible range up to 400~meters and throughput up to 2~Mbps over Bluetooth link itself~\citep{Bluetooth2016bluetooth}.

While Infrared requires line of sight, Bluetooth can communicate over barriers due to the radio frequency property.
It is possible to have many devices interconnected through a master device or diverse networks interconnected through the slaves, and, additionally, current specification improves broadcasting and low energy communication.
Bluetooth 3.0 High Speed can also exchange data up to 24~Mbps using Bluetooth link to establish the connection and finally communicating through a secondary radio available on devices, such as 5~GHz radio used for WiFi~\citep{Bluetooth2009bluetooth}.
Although, the Bluetooth Low Energy approach is getting accepted as reliable for many solutions that requires to transfer just few data from time to time, the relation between power consumption and data exchange is still the main drawback of Bluetooth technology.

The NFC shows up as another option for mobile device communication with an one to one network topology and fast setup time~\citep{Coskun2013nfc}.
Developed in 2002, the technology aims fast contactless communication in a range of 4 to 10~cm with a data rate up to 0.4~Mbps.
Although it is possible to have two mobile devices communication through NFC, in most of the situations the data exchanging is made from an active device that reads a passive NFC Tag and do ``something'' with the data read afterwards, such as open a website or confirm a payment.

The WiFi technology has many options for short range communication.
Devices can interconnect through a home router and exchange data using the Local Area Network~(LAN) using wired or wireless connection.
In case of wired connection, the Ethernet network will depend on the available technology on all devices that currently varies from 10 to 1000~Mbps.

The wireless technology have specific details that depend on the antenna, router, and network structure.
WiFi antennas have its power defined in decibels-isotropic~(dBi) and the the transmission can be directional or omni-directional.
The router needs to be compatible to give sufficient power to the antenna in order to use the full gain available.
Additionally, the router can support different transmission standards, such as IEEE 802.11 a, b, g, n, ac, and many others.
It is necessary to notice that two devices need to share the same standard in order to communicate, and this cause the industry to sell wireless routers compatible with as many standards as possible.
The network structure can have router, switch, hub, repeater, and many other devices aimed to interconnect hosts.
In terms of structure, the possibilities are infinite as long as one network can have its particular structure connected to other network defined in another particular structure.

An optional use of WiFi for short range interconnection is the adoption of WiFi Direct~\citep{Alliance2010wifii}.
As the WiFi direct is mostly used by smartphones, the power of the antennas can offer a maximum range of 200~m and the network interfaces currently offer 250~Mbps~\citep{Feng2014d2d} maximum throughput, even if some new devices can have gigabit WiFi interfaces.
The WiFi direct technology works like the hotspot, which is an option available in most devices that creates a network using the smartphone as a router.
WiFi Direct is a peer to peer~(P2P) connection and allows two devices to exchange data like in other short range solutions but using WiFi technology.

While there are many new short range technologies that works with dozen of meters range, they are designed for interconnecting devices close to each other.
A short range technology can work for long range communication if one of the interfaces connects with another interface that would connect with another interface and so on.
%%FEULO ...another innterface that would connect to a third interface nd so on.
In this case, the common setup is to have a device connected through a short range technology and another device connected to the Internet.
This situation keeps the users close to a short range interface but communicating with devices far away, what may be undesirable in some cases, and it is similar to the use o telephones instead of mobile phones. 

\subsection*{Long range technologies} 

\paragraphdesc{wired and wireless}
For some short range technologies, the use may be similar to be connected to long cables at a restricted area, even being wireless.
%%FEULO eh short range aqii mesmo??
Wired connections would be most suitable in some situations where the devices do have to stay in a small space or fixed place.
The quantity of devices connected through wired connection is also another problem as the cable needs space.
On the other hand, wireless connections would offer the advantage of being free of cables and allow users to keep moving freely while communicating.

\paragraphdesc{long range interfaces descriptions}
Long range communication require specific network structure or interface: a short range interface can be used to communicate with an interface connected to the Internet or long range interface can communicate with an antenna far away.
Although short range interfaces can be used for long range communication, they are not projected for this aim and they can consume more power in this situation.
The most common long range interfaces are WiFi and interfaces used for mobile telecommunication such as 3G or 4G.
These interfaces provide fast data exchanging and allow communication with antennas far away geographically. 

\paragraphdesc{WiFi}
WiFi technology works well as short and long range, and although the power consumption is almost 2 times higher than some short range alternative such as Bluetooth~\citep{Friedman2013wifibluetooth}, the advantage is the throughput that can reaches gigabits per seconds in some new standards and the range that can reaches dozen of kilometers with some specific antennas~\citep{Raman2007Experienceswifiindia}.
Short range technologies are also restrictive due to the number of connected devices, while long range technologies can have a hundred devices connected to the same antenna at the same time.
A wireless router for WiFi connection can have limitations regarding the number of connected devices even if most of them are able to offer more then 200 IP addresses.
These limitations depend on the routing algorithm and the antenna power that are technical settings related to device's model and brand.

The IEEE 802.11 standards defines the possible WiFi technologies that can be available on any device, such as 802.11 a, b, g, n, and ac standards
These standards are related to the network physical layer and they are available at IEEE Standards Association~(IEEE-SA) website~\footnote{IEEE Standards Association website: \url{http://standards.ieee.org/}}.
A WiFi router is able to implement different standards at the same time in order to be compatible with as many device as possible, but the router needs to offer a high power to its antenna in order to have a long range signal strength and allow devices to move far away.

\paragraphdesc{4G}
Another network interface available in mobile devices is conceived for mobile telecommunications.
2G, 3G, 4G, or 5G are current options for mobile telecommunication.
The name of these technologies are related to their generation, represented by the number in front of the `G' which stands for generation.
%%FEULO fico meio repetitivo esse role d repetição
A new generation of mobile telecommunication technology is defined when a new standard is defined without back compatibility.
A device compatible with 4G is able to connect to the older generations only when the manufacture implements the compatibility so that the device can switch between all of them, however the network interface of this device will dismiss any information regarding a new generation such as 5G.
2G has a power consumption higher than 3G and 3G consumes more power than WiFi, though it is expected that a new generation will always offer higher throughput and less power consumption than older ones~\citep{Rice2010decomposing,Balasubramanian2009energy}.

The main advantage of these mobile telecommunication technologies are the range and mobility.
Tall and powerful antennas are spread throughout the cities providing signal in every open space, on the other hand the devices can switch between these antennas transparently to the user.
This characteristic provides more mobility for this technology without the burden of establishing connection or depending on extra settings or devices.

\paragraphdesc{Network interfaces and Internet Access}
While connected to the Internet, the devices keep trying to switch to the best network interface for exchanging data in the area and saving costs.
The costs here are related both to money expenses on data plan and energy consumed while communicating.
It is expected that devices will switch between mobile data to WiFi when possible, and from WiFi to any wired connection as well.
The network topology is also important at this point, as the device may be unaware of which technology has internet access and in this case particular setups are required for specific purposes.


















\section{Barriers in long distance networking}

Aiming music interaction, the long distance networking seems to be a difficult conception.
When trying to play music with performers connected via long distance networks through mobile devices, the main barrier is the network latency in the communication between devices from different locations.
Another barrier is the setup necessary to permit the interconnection between these devices in these networks through a wireless connection. 
In terms of multimedia applications, the latency need to be inferior to 150ms in order to looks like a synchronous interaction~\citep{Coulouris2011distributed}.
A latency of 100ms and above implies some difficulty for network interaction even with experienced musicians~\citep{Bartlette2006networkeffect}, and if we have a sensitive ensemble performance, this threshold may be as low as 20ms~\citep{Chafe2004network}.
\cite{Lago2004thequest} says that 50ms is acceptable for chamber music, but the range between 20ms to 30ms is probable the perfect latency for remote musical applications~\citep{Lago2004thequest}. 

The latency has a lower limit based on some constraints.
If we consider the speed of light on optical fiber~(200,000km/s) as a reference for data transmission and Earth circumference around Equator~(40,000km) as the longest distance, a packet would take 200ms to finish the round trip.
This result implies a latency of 100ms and it shows that interactions from different countries will have problems.
One way to tackle this problem in local networks is using lightweight transport protocols such as UDP~\citep{Harker2008laptop,Caceres2010jacktrip}, that has a low overhead when compared with TCP and fast data diffusion because it does not need the establishment of connections and acknowledgment to each packet.
In case we need to communicate with many devices at the same time, UDP has another advantage: it can be used in Multicast communication that we are going to discuss in Section \ref{sec:multicast}.   

The disadvantage of UDP is the possibility of packet loss during transmission.
Applications that rely on UDP communication need some alternatives to take on this condition.
The order of the packets may not be guaranteed in this communication as well and in some cases a missing packet is just a delayed packet that may can be discarded or not.
These problems are some characteristics of UDP communication and they will probably occur more frequently on long distance connections than in a small local network.
However, the setup of UDP is nearly network independent in case of a local or big network.
A big network with the same services available in a local network is a suitable environment for applying solutions that were previously proposed for local networks respecting the distance constraints and the distribution of the devices to be configured when necessary.
Academic institutions are connected most of the time through a research network that has direct connection between members and offers access to the Internet as well.
The research networks from different countries are usually interconnected and they have support for network experiments when requested.
The bandwidth of these networks are normally higher than popular Internet connections due to its research and experimental purpose.

An advantage of these networks for long distance networking is that their structure is well defined and they can be used as local networks with public IPv4 addresses and many services that are only available in local networks.
One of these services is the Multicast and we are going to discuss in the next section.
Another barrier that is impossible to predict the limits is the setup necessary to start using these networks.
In local networks we have normally one router and devices that are easily accessible in a physical and technical manner.
Academic networks are managed by many different partners and each point has its own rules and definitions.

The configuration of the services in each network depends on their own managers.
The routers have changes restriction for security reasons, as most the routers are responsible for a huge data traffic from many universities and they need to avoid network attacks.
The communication between the managers during a network setup is essential in order to minimize the risk of problems regarding a wrong configuration at some point.
In the section below we will discuss the Multicast focusing on the restrictions its use and setup inside academic networks.

\section{Multicast}
\label{sec:multicast}

We have different network methods for exchanging messages using UDP.
The Unicast is used to send messages to one and only one point on the network, while the Broadcast can be used to diffuse messages to all connected devices inside a network.  
Multicast permits packet diffusion to a group of devices that had already subscribed to the group.

Multicast is available through various multicasting mechanisms, e.g. IP Multicast~\citep{Diot2000ipmulticast}.
Additionally, the network need to understand a protocol like Multicast Source Discovery Protocol (MSDP) in order to permit devices to find Multicast groups inside the network and create routes between them through the Randevouz points~(RP).
The RPs defined on a network are the routers that serve as an encounter point from the information regarding group announcements and device's join requests.

Routers available in most of the houses have been configured with Multicast nativelly and the users don't need to change anything.
On the other hand, the same is not true at academic networks.
The routers available at most institutions on academic networks have specific hardware and software configuration due to the high number of computers connected through them.
It implies many security settings and filters that may interfere with the use of Multicast.
Although some routers come with the support to Multicast, their default settings avoid announcement of Multicast groups to devices outside the local network and the router will discard announcements arriving from outside.
The devices can use the method internally but will not be able to communicate with devices at other institutions through Multicast.

Network administrators can enable the MSDP at these routers when the protocol is supported by the network interface on the router.
In this case, all devices in the route need to accept the same configuration.
Routers or network interfaces may or may not support the necessary settings, and in the last case they need to be changed or the route need to follow another path to complete the tree of nodes.    

At this point the users need to select and join a Multicast group before using it.
This group is defined by an IP and port.
The IP range for Multicast is defined on RFC1112~\citep{RFC1112Multicasting} and goes from 224.0.0.0 to 239.255.255.255.
The port number is expected to be an unassigned port in order to avoid problems with any other running service.
The RFC6335~\citep{RFC6335IANA} suggests the use of any dynamic port from 49152 to 65535 during tests or evaluations.
The dynamic ports will never be assigned to any specific software by IANA~\footnote{Internet Assigned Numbers Authority (IANA): \url{http://www.iana.org}} and the ports are expected to be free in most part of time.

Other network solutions for distributed communication are available for situations outside local or private networks.
One of these technologies is the Cloud Computing and the Cloud Services, which are discussed in the next section.



















%% ------------------------------------------------------------------------- %%
%\section{Network Topology}\index{mobile networks!network topologies}
%\label{sec:topologies}


%TODO: discuss peer-to-peer and other computer network structures (ring, star, cluster, grid, cloud..)
\paragraphdesc{standard topology: ring, star, cluster, grid}

\paragraphdesc{one-to-one, one-to-many, many-to-many communication: p2p, server, cloud}
% Direct communication or one-to-one data transmission is a common setting for mobile network technologies, but we can also take advantage of one-to-many or many-to-many.
% The structure of a network is an important definition when we are planning the distribution of data between deices during a musical performance as we need to define the technologies to be used based on their availability.
% The resultant delay for exchanging messages may also depend on the network structure and will probably affect some performances.


\paragraphdesc{topology for computer music: Alvaro, Weinberg, Lee, Carot}
%TODO: discuss Alvaro and Weinberg works

\paragraphdesc{Alvaro discussion}

\paragraphdesc{Weinberg diagrams}

%TODO: discuss the work from Lee and Essl
\paragraphdesc{Lee revisiting Weinberg}
% Another recent work revisits these structures and evaluate the possible structures that require some attention to be improved and experimented:

\paragraphdesc{Carot models}

\paragraphdesc{aesthetic discussion regarding topology}

\paragraphdesc{when topology depends on technology}

%% ------------------------------------------------------------------------- %%
%\section{Protocols}\index{mobile networks!protocols}
%\label{sec:protocols}

\paragraphdesc{communication channels, code, and translations}
% In order to communicate, mobile devices need a connection between each other and use some protocols.
% The connection depends on the network interface and we need protocols to ensure that different interfaces communicate with each other.
% Both endpoints can use different connections and protocols as long as we have some commuter at some point in the path between them.
% A device can use Bluetooth to send data to a computer that will send this data through Wifi to another device that is going to receive the data from the WiFi and send to another device using infrared, for example, and than we are using different interfaces and protocols.
% Network structures define the way devices can communicate but the connection used may be specified in many ways.
% Some interfaces like infrared require the sender and receiver to share the same medium and avoid barriers while WiFi requires address and specific network settings to allow communication.
% We can also control one or many televisions with one single infrared remote control and send a message to one or many devices through the same WiFi connection.

\paragraphdesc{device addressing: local and internet}
% The device addressing is related with the network interface.
% In case of wireless connection, the frequency is important and some wireless interfaces share an identification attached to the interface like in Bluetooth, WiFi, and mobile networks.
% Some wireless interfaces also include name tags or other kind of address like IP, in order to be reached.

\paragraphdesc{network protocols: IP, TCP, UDP}
% When the interfaces are connected to the Internet using the IP, they also make use of specific protocols including TCP and UDP.
% These protocols differ in the way the messages are exchanged between devices and may be suitable only to some networks structures.

%TODO: discuss about IP, TCP, UDP... cloud

\paragraphdesc{TCP}

\paragraphdesc{UDP}

\paragraphdesc{cloud options: tcp keep alive and web sockets}

\paragraphdesc{protocols and mobility}

%% ------------------------------------------------------------------------- %%
\section{Cloud Services}\index{mobile networks!cloud services}
\label{sec:cloudservices}

\paragraphdesc{cloud computing scalability}
Once the devices are often connected to the Internet, other technologies become available to implement network structures.
Peer-to-peer, Ajax, sockets, web services, and many other solutions have been used to allow devices to exchange data.
Although most of them present good performance, the advances on Cloud Computing provided support for many devices interconnection with on demand improvements regarding the capacity of services.

\paragraphdesc{clod computing solutions}
The implementation of services on the cloud is available by many companies like Amazon, Microsoft, Intel, IBM, and Google.
On top of that we have companies that offer services already implemented on the cloud, make them ready to use, and take advantage of cloud features at the same time.
The cloud services offered by these companies are commonly used today due to their facilities and performance.
As the users have to use the services without access to the implementation, the evaluation of many cloud services may help to decide which one suits their needs.

\paragraphdesc{cloud services}
Cloud services are services that are deployed on a cloud computing structure to take advantage of its computation and distribution qualities.
Cloud Services differs in terms of message size, limit of connected devices, messages per second, and location of servers.
Although these settings may be default in most cases, the companies have different plans that expand the possibilities and can also extend the capacities on demand depending on the situation.
A common use is the data replication service at websites and mobile applications, so even if we have increasing access at some moment, the cloud service can instantiate another machine or machines to provide minimum latency and avoid processing overhead on the servers.

During this research some applications were developed using two specific cloud services: Pusher and PubNub.
These cloud services were selected due to their popularity in the time of this research.
Although they are quite similar, they implement specific rules to use their APIs that can interfere in some interaction approaches.

%TODO: discuss about Cloud Computing (Amazon), Pusher, PubNub, JSON
\subsection*{Pusher cloud service}
\paragraphdesc{Pusher}

Numerous mobile applications have a focus on fast message delivery to a high number of devices in many parts of the world.
A good example of this service is an email application that sends us a notification whenever we receive a new message without requiring us to actively request new information.
This approach is called push notifications, and one implementation of it is Pusher.
This cloud service uses cloud computing solutions in order to offer a service that delivers messages through web sockets and HTTP streaming.

One of the advantageous features supported by Pusher API is its support of HTTP Keep-Alive feature. 
Once connected to Pusher cloud service, it is possible to send and receive messages to/from the cluster without starting a new connection, saving the overhead of restarting new TCP connections.
The Pusher service offers the possibility of creating real-time applications considering all of its resources.

Although Pusher has many paid plans with more resources available, a free plan is also available for use.
The free plan has a limit of 100,000 messages per day and it counts the API requests and messages delivered to each client.
We can have up to 20 clients connected at the same time and all clients will send and receive messages only from the US-East cluster server that is situated in Northern Virginia.

The service also has some restrictions for all plans, free and paid alike.
Clients are allowed to send no more that 10 messages per second and will be disconnected from the service when they exceed this limit.
This limitation is due to the overhead on distributing messages among a thousand users.
Every message has a size limit of 10 kilobytes, but it is possible to request an upgrade in order to send larger messages.

The requirement to exchange messages between users is to create a channel and have users connected to the same channel.
The API supports public, private, and presence channels.
Public channels are used only to send messages from the server to the users connected, like a feed.
Private channels need a prefix ``private-'', require server authentication, and offer the option to accept messages from its users.
The presence channel is similar to the private channel, and includes the feature of requesting information about connected users. 
The channels may have different events that can be bound by clients, e.g. a client can bind the event ``client-event'' and receive notifications when a new event with the same name is sent to the channel.
On private and presence channels, client events must have the ``client-'' prefix.
An example code used to send and receive messages through the Pusher cloud service using a private channel is presented on Listing~\ref{JavaPusherAPI}.

\begin{footnotesize}
\lstset{language=Java, caption=Example of Java code from Pusher API, captionpos=b, label=JavaPusherAPI, numbers=none, numberstyle=\scriptsize}
\begin{lstlisting}[frame=single]
Pusher pusher;
PrivateChannel channel;

HttpAuthorizer authorizer = 
	new HttpAuthorizer(AUTH_PAGE);
PusherOptions options = new PusherOptions();
options.setAuthorizer(authorizer);
pusher = new Pusher(PUSHER_API_KEY, options);
pusher.connect();

String channelName = "private-channel";
channel = pusher.subscribePrivate(channelName);
String eventName = "client-event";
channel.bind(eventName, 
	new PrivateChannelEventListener() {...});
JSONObject jsonObject = new JSONObject();
String message = MESSAGE;
jsonObject.put("message", message);
channel.trigger(eventName, jsonObject.toString());
\end{lstlisting}
\end{footnotesize}

In this piece of code, there are information that depends on the application settings.
The first is the ``AUTH\_PAGE'', that needs to be configured to give a unique authentication code to every socket connection.
The ``PUSHER\_API\_KEY'' is the key received when creating an application on Pusher.
After creating the channel, the programmer needs to bind an event with an event listener.
The event listener requires the code that is expected to run every time the event occurs.
The ``MESSAGE'' is a string with any values defined to be sent, e.g. numbers need to be converted directly to string or through Base64 binary-to-text encoding schema.

\subsection*{PubNub}

Although similar to Pusher, PubNub presents some differences that intimated its use in this research just for comparison reasons.
PubNub cloud service provides an extensive API and SDKs, and the service can be easily configured with a few functions.
Initially, the application needs to get an UUID and initialize the PubNub object including information regarding the account: the publish key and the subscribe key.
After that, the device can publish messages to any channel as long as the channel name is known (e.g. ``performer'', ``audience'').
When subscribing to a channel, it is also necessary to define a callback function that will typically parse the received messages.
The basic function to set up a publish-subscribe mechanism is presented on Listing~\ref{JavaScriptPubNubAPI}.

\lstset{language=bash,  caption=Example of JavaScript code from PubNub API presented at audience page, captionpos=b, label=JavaScriptPubNubAPI, numbers=none, numberstyle=\scriptsize}

\begin{lstlisting}[frame=single, float=t]
// Request an UUID
var my_id = PUBNUB.uuid();
// Initialize with Publish & Subscribe Keys
var pubnub = PUBNUB.init({
 publish_key: publishKey,
 subscribe_key: subscribeKey,
 uuid: my_id,
});

// Subscribe to a channel
pubnub.subscribe({
 channel: my_id + ",audience",
 message: parseMessage, // callback for msg
 error: function (error) {
  // Handle error here
 },
 heartbeat: 15
});

// Parse received message
function parseMessage( message ) {
 if (typeof message.type !== 'undefined'){
  if ( message.type == "create-response"){
   // Do something
  }
 } else if ...
}

// Publish a message to a channel
pubnub.publish({
 channel: "performer",
 message: {"type":"create", 
  "my_id":my_id,
  "nickname": strScreenName},
 error : function(m) {
  // Handle error here
 }
});
\end{lstlisting}

The advantages of PubNub, compared to Pusher, include larger message size and a greater number of messages allowed. 
Both services have plans that preserve the size of the messages but differ in the allowable quantity of serviced messages.
While Pusher limits the messages to 10KB, PubNub accepts messages of 32KB.
Another disadvantage of Pusher, especially in the context of audience participation music, is that it will discard messages if the device exceeds the rate of 10 messages per second and plans limit the number of messages sent per day.
PubNub limits the quantity of messages sent per month but the limits are never throttled even on free plans, so the users can use the full service capacity during one single performance, assuming no other performance within that month.

Pusher and PubNub can send any type of data through the cloud respecting plan limits.
For instance, one might share codes from computer music languages like CSound\footnote{CSound: \url{http://www.csounds.com/}}, ChucK\footnote{ChucK: \url{http://chuck.cs.princeton.edu/}}, and SuperCollider\footnote{SuperCollider: \url{http://supercollider.github.io/}}.
Cloud services can also share symbolic data or control signals between any mobile applications, allowing one to make use of any computer music synthesis engines available in tandem with Cloud services.
